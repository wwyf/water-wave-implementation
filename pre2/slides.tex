%%%%%%%%%%%%%%%%%%%%%%%%%%%%%%%%%%%%%%%%%%%%%%%%%%%%%%%%%%%%%%%%%%%%%%%%%%%%%%%%%%%%%%%%%%%%%%
% Template Beamer Sugestivo para Projetos no Senac
% by ezefranca.com
% Based on MIT Beamer Template
% As cores laranja e azul seguem o padrao proposto no manual de uso da identidade visual senac
%%%%%%%%%%%%%%%%%%%%%%%%%%%%%%%%%%%%%%%%%%%%%%%%%%%%%%%%%%%%%%%%%%%%%%%%%%%%%%%%%%%%%%%%%%%%%% 

%\documentclass{beamer} %voce pode usar este modelo tambem
\documentclass[handout,t]{beamer}
\usepackage{graphicx,url}
\usepackage[english]{babel}   
\usepackage{ctex}
\usepackage[utf8]{inputenc}
\batchmode
% \usepackage{pgfpages}
% \pgfpagesuselayout{4 on 1}[letterpaper,landscape,border shrink=5mm]
\usepackage{amsmath,amssymb,enumerate,epsfig,bbm,calc,color,ifthen,capt-of, multimedia, hyperref}
\usetheme{Berlin}
\usecolortheme{sustech}
\usepackage{caption}
\usepackage{subcaption}
\usepackage{graphicx}
\usepackage{color}
\usepackage{amssymb}

%-------------------------Titulo/Autores/Orientador------------------------------------------------
\title{Realistic Water Simulation}
\author{Team Wavy}
\institute{School of Data and Computer Science(SYSU)}
\date{\today}


%-------------------------Logo na parte de baixo do slide------------------------------------------
\pgfdeclareimage[height=0.7cm]{sustech-logo}{logo.png}
\logo{\pgfuseimage{sustech-logo}\hspace*{0.5cm}}

%-------------------------Este código faz o menuzinho bacana na parte superior do slide------------
\AtBeginSection[]
{
  \begin{frame}<beamer>
    \frametitle{Outline}
    \tableofcontents[currentsection]
  \end{frame}
}
\beamerdefaultoverlayspecification{<+->}
% -----------------------------------------------------------------------------
\begin{document}
% -----------------------------------------------------------------------------

%---Gerador de Sumário---------------------------------------------------------
\frame{\titlepage}
\section[]{}
\begin{frame}{Content}
  \tableofcontents
\end{frame}
%---Fim do Sumário------------------------------------------------------------

\section{Introduction}
\subsection{Preface}
\begin{frame}[t]{\emph{HOW} to simulate?}
  \textcolor[rgb]{0.5,0.5,0.5}{Quite difficult to deal with the \textbf{\color{blue}volatile} material...}
  \begin{columns} % align columns
    \begin{column}{.48\textwidth}
      \begin{figure}[thpb]
        \centering
        \resizebox{0.8\linewidth}{!}{
        \includegraphics{figures/pic6.jpg}
        }
        %\includegraphics[scale=1.0]{figurefile}
        %\caption*{System components}
        \label{fig:system}
      \end{figure}
    \end{column}%
    \hfill%
    \begin{column}{.48\textwidth}
      \begin{itemize}
      \setbeamertemplate{itemize items}{\color{red}$\bullet$}  
        \item Texture-based method : minimize the calculation, wildly used in real time rendering
          \begin{itemize}
            \setbeamertemplate{items}[ball]  
            \item Blinn, 1978, Bump Mapping
          \end{itemize}
        \item Construction-based method : more mathematicall
          \begin{itemize}
            \setbeamertemplate{items}[ball]  
            \item Cosine function superposition algorithm
            \item Gerstner Wave(\color{red} pick it!\checkmark)
            \item B-spline
          \end{itemize}
      \end{itemize}
    \end{column}%
  \end{columns}
\end{frame}

\begin{frame}[t]{\emph{HOW} to simulate?}
  \textcolor[rgb]{0.5,0.5,0.5}{Quite difficult to deal with the \textbf{\color{blue}volatile} material...}
  \begin{columns} % align columns
    \begin{column}<0->{.48\textwidth}
      \begin{figure}[thpb]
        \centering
        \resizebox{0.8\linewidth}{!}{
        \includegraphics{figures/pic6.jpg}
        }
        %\includegraphics[scale=1.0]{figurefile}
        %\caption*{System components}
        \label{fig:system}
      \end{figure}
    \end{column}%
    \hfill%
    \begin{column}<0->{.58\textwidth}
      \\
      \begin{itemize}
      \setbeamertemplate{itemize items}{\color{red}$\bullet$}  
        \item Based on physics models : Realistic and Lifelike
        \begin{figure}[thpb]
          \centering
          \resizebox{0.8\linewidth}{!}{
              \centering
              \includegraphics{figures/pic8.png}
          }
        \label{fig:system}
        \end{figure}
      \end{itemize}
    \end{column}%
  \end{columns}
\end{frame}


\section{Implementation}

\subsection{Gerstner Wave Implementation}

\begin{frame}{Introduction to Gerstner Wave}
  \begin{itemize}
    \item Wavelength (L)\\ 
    %the crest-to-crest distance between waves in world space. 
    %Wavelength L relates to frequency w as $w = 2/L$.
    \item Amplitude (A)\\
    %the height from the water plane to the wave crest.
    \item Speed (S)\\
    %the distance the crest moves forward per second. 
    %It is convenient to express speed as phase-constant phase-constant.jpg , where phase-constant.jpg = S x 2/L.
    \item Direction (D )\\
    %the horizontal vector perpendicular to the wave front 
    %along which the crest travels.
  \end{itemize}
  \begin{figure}[thpb]
    \centering
    \resizebox{0.6\linewidth}{!}{
        \centering
        \includegraphics{figures/gerstner_intro.jpg}
    }
  \label{fig:gerstner_intro}
  \end{figure}
\end{frame}

\begin{frame}{Introduction to Gerstner Wave}
  对于单个波而言,其函数表达式如下所示。\\ 
  \begin{itemize}
    \item A 振幅 \\
    \item D 二维方向向量 \\
    \item w 角速度 \\
    \item $\varphi$ 其值为$S \times \frac{2}{L}$
  \end{itemize}
  $$W_i(x,y,t) = A_i \times \sin(D_i \cdot (x,y) \times w_i + t \times \varphi_i)$$
\end{frame}

\begin{frame}{Introduction to Gerstner Wave}
  完整的Gerstner Wave函数如下所示。其为向量值函数。
  \begin{itemize}
    \item 输入\\
      网格化的坐标值\\ 
    \item 输入 \\
      三维坐标系中的坐标(x, y, z) \\
  \end{itemize}
  $$
  P(x, y, t) = 
    \begin{pmatrix}
    x + \sum(Q_iA_i \times D_i[x] \times \cos(w_iD_i \cdot(x, y) + \varphi_it))\\
    y + \sum(Q_iA_i \times D_i[y] \times \cos(w_iD_i \cdot(x, y) + \varphi_it))\\
    \sum(A_i\sin(w_iD_i \cdot(x,y) + \varphi_i t))
    \end{pmatrix}
  $$
\end{frame}

\begin{frame}{参数特点}
  \begin{itemize}
    \item Q较大 \\
    波峰尖锐,类似于真实水面 \\
    \item Q过大 \\
    产生环,破坏波的形状 \\
    \item Q为0 \\ 
    退化为简单三角函数的叠加 \\
  \end{itemize}

  \begin{figure}[thpb]
    \centering
    \resizebox{1\linewidth}{!}{
        %\includegraphics{figures/pic1.png}
        \begin{subfigure}[t]{0.5\textwidth}
        \centering
        \includegraphics[width=1\textwidth]{figures/loop.jpg}
        \subcaption*{The first film}
        \end{subfigure}
        \quad
        \begin{subfigure}[t]{0.5\textwidth}
        \centering
        \includegraphics[width=1\textwidth]{figures/sharp.jpg}
        \subcaption*{The second film}
        \end{subfigure}
    }
    %\caption*{\emph{Finding Nemo}({\fangsong 海底总动员})}
  %\label{fig:1}
  \end{figure}
\end{frame}


\begin{frame}{Advantages}
  \begin{itemize}
    \item 波峰尖锐,波谷平缓
    \item 允许多波产生叠加效果,交互感真实
    \item 参数众多,可调节性强
    \item 性能较高,开销较小
  \end{itemize}
  \begin{figure}[thpb]
    \centering
    \resizebox{0.5\linewidth}{!}{
        \centering
        \includegraphics{figures/gerstner_dem.jpg}
    }
  \caption{Gerstner Wave}\label{fig:gerstner_dem}
  \end{figure}
\end{frame}

\begin{frame}{Disadvantaged}
  \begin{itemize}
    \item 水面永动,水面远处永远“自发地”泛起波纹
    \item 波峰低的波会产生大面积规律褶皱
    \item 无法仿真不同向的波间的碰撞和能量损耗。
        当波间相差的角度较大时,水面效果较差
    \item 整体上看,规律性过强
  \end{itemize}
\end{frame}


\begin{frame}{Refine}
  \begin{itemize}
    \item 波随机过滤\\
    对小波采取随机化地去除。
    \item 动态修改参数\\
    动态地略微地修改各个波的方向、振幅等参数。
  \end{itemize}

  \begin{figure}[thpb]
    \centering
    \resizebox{0.4\linewidth}{!}{
        \centering
        \includegraphics{figures/show.png}
    }
  \caption{程序运行图片}\label{fig:show}
  \end{figure}


\end{frame}


\section{Continue to Implementation}

\subsection{Our Implementation}


\begin{frame}{Three ways to implementation}
  
water wave packet论文中提到了三种方法用来实现水面
  
  \begin{itemize}
    \item 无限的波列(infinitely long wavetrains)
    \item 单波峰(single wave crests)
    \item 波包(packets of waves)
  \end{itemize}
  
我们自己根据 Gerstner 波的相关知识,使用无限的波列渲染了简单的水面。
  
后来,论文的思想,进一步启发了我们使用单波峰完善了水面的渲染。

最后尝试了波包在水面渲染上的应用。

\end{frame}

\begin{frame}{实现效果}

  \begin{columns} % align columns
    \begin{column}<0->{.5\textwidth}
      \begin{figure}[thpb]
        \centering
        \resizebox{0.8\linewidth}{!}{
        \includegraphics{figure/2018-07-03-01-41-41.png}
        }
        %\includegraphics[scale=1.0]{figurefile}
        \caption{单个波浪}
        \label{fig:wyf-show-effect-left}
      \end{figure}
    \end{column}%
    \hfill%
    \begin{column}<0->{.5\textwidth}      
      \begin{figure}[thpb]
        \centering
        \resizebox{0.8\linewidth}{!}{
          \includegraphics{figure/2018-07-03-01-51-46.png}
          }
          %\includegraphics[scale=1.0]{figurefile}
          %\caption*{System components}
          \label{fig:wyf-show-effect-right}
          \caption{多个波浪}
      \end{figure}
    \end{column}%
  \end{columns}

\end{frame}


\subsection{single wave crests}

\begin{frame}

我们使用单波峰实现了对水面扰动的仿真,具体实现特性有以下几点:

\begin{itemize}
  \item 按一次键盘->扰动->波浪
  \item 扰动强弱的控制
  \item 水面扰动能量的衰减
  \item 多个扰动间的叠加
\end{itemize}


\begin{columns} % align columns
  \begin{column}<0->{.5\textwidth}
    \begin{figure}[thpb]
      \centering
      \resizebox{0.7\linewidth}{!}{
      \includegraphics{figure/2018-07-03-01-59-35.png}
      }
      %\includegraphics[scale=1.0]{figurefile}
      \caption{单个扰动刚出现}
      \label{fig:wyf-single-left}
    \end{figure}
  \end{column}%
  \hfill%
  \begin{column}<0->{.5\textwidth}      
    \begin{figure}[thpb]
      \centering
      \resizebox{0.7\linewidth}{!}{
        \includegraphics{figure/2018-07-03-01-59-47.png}
        }
        %\includegraphics[scale=1.0]{figurefile}
        %\caption*{System components}
        \label{fig:wyf-single-right}
        \caption{扰动接近消失}
    \end{figure}
  \end{column}%
\end{columns}


\end{frame}


\begin{frame}{大致实现思路}
  
将波独立成波峰或者波包来实现

\begin{itemize}
  \item 对波的控制更为细致
  \item 易于实现能量衰减
  \item 不同波之间的相互影响有着成熟的理论体系
\end{itemize}

使用波包来对水面进行模拟,能够基于物理知识,渲染出更加真实且效果更佳丰富的水面。
% \\

在进一步的实现中,我们仅使用了单波峰来完善我们的水面的渲染。

  
\end{frame}

\begin{frame}{具体实现思路}

\begin{itemize}
  \item 实现了两个类
  \begin{itemize}
    \item Packet类
    \item PacketManager类
  \end{itemize}
\end{itemize}


\begin{columns} % align columns
  \begin{column}<0->{.5\textwidth}
    Packet类
    \begin{itemize}
      \item 存有该波峰的能量,波长,振幅,起始坐标等
      \item 存放于PacketManager类中的vector容器中
      \item 用户不直接使用Packet
    \end{itemize}
  \end{column}%
  \hfill%
  \begin{column}<0->{.5\textwidth}      
    PacketManager类
    \begin{itemize}
      \item 管理该水面具有的Packet
      \item 增加新的Packet
      \item 遍历vector容器,更新当前水面信息
      \item 检测到波峰能量为0时,将该波峰从容器中删除
    \end{itemize}
  \end{column}%
\end{columns}
  
\end{frame}

\begin{frame}{波高信息的更新}

在每一帧,程序的主渲染循环都会调用PacketManager的\texttt{update\_data}接口。\\

正是这关键的操作,实现了波峰的移动及消失

\begin{figure}[thpb]
  \centering
  \resizebox{0.7\linewidth}{!}{
    \includegraphics{figure/figure/2018-07-03-01-59-47.png.png}
    }
    %\includegraphics[scale=1.0]{figurefile}
    %\caption*{System components}
    \label{fig:wyf-code}
    \caption{一处关键代码}
\end{figure}

\end{frame}

\begin{frame}{不足与优点}
  
在我们的实现中,我们认为有如下优点。

\begin{itemize}
  \item 不使用波序列,而使用单个的波峰进行模拟
  \item 能量衰减实现效果自然
\end{itemize}
    
不过,仍有以下不足。

\begin{itemize}
  \item 物理背景缺乏,较难实现与刚体碰撞后波的进一步反弹与传播
  \item 单波峰扰动效果一般,波峰过于圆润,不够真实
\end{itemize}

    
\end{frame}



% -----------------------------------------------------------------------------
\section{A fun demo}
\subsection{A fun demo}
\begin{frame}{Demo}
  \centering
  \Huge \textbf{\textcolor[rgb]{0,0,1}{Now is the demo time!}}
\end{frame}

\begin{frame}{Group member}
 \large
  \begin{center}
    \begin{tabular}{cc}%
    李新锐\ 15323032&孙一言\ 16337216\\
    王锦鹏\ 16337232&王永锋\ 16337237\\
    颜彬\ 16337269&韦博耀\ 16337242
    \end{tabular}
  \end{center}
\end{frame}

\begin{frame}{Thank you}
  \begin{center}
    \Huge Thank you for listening!
  \end{center}
\end{frame}



% \subsection{packet wave}


% \subsection{Conclusion}

% % -----------------------------------------------------------------------------
% \section{Introducão}
% \begin{frame}{Introdução}
% %introducao
% A Introdução vai aqui
% \end{frame}
% %------------------------------------------------------------------------------

% %------------------------------------------------------------------------------
% \section{Referencial Teórico}
% \begin{frame}{Referencial Teórico}
% %referencial teorico, estado da arte, etc
% Referencial Teórico ou estado da arte.  
% \end{frame}
% %------------------------------------------------------------------------------

% %------------------------------------------------------------------------------
% \section{Metodologia}
% \begin{frame}{Metodologia}
% Metodologia minuciosamente aqui.
% \end{frame}
% %------------------------------------------------------------------------------

% %------------------------------------------------------------------------------
% \section{Considerações e Resultados}
% \begin{frame}{Considerações e Resultados}
% %consideraçoes e resultados
% Resultados e considerações do trabalho  
% \end{frame}
% %------------------------------------------------------------------------------

% %------------------------------------------------------------------------------
% \section{Referencias}
% \begin{frame}{Referencias}
%   Suas referencias bibliográficas aqui, siga o modelo ABNT.
% \end{frame}
% %------------------------------------------------------------------------------

% %------------------------------------------------------------------------------
% \section{Agradecimentos}
% \begin{frame}{Agradecimentos}
% Agradeço a todos que colaboraram  para realizaç~ao deste projeto.
% Agradeço ao Alexandre Bencz pelo assembly de cada dia.
% Agradeço ao Sonata Arctica, Avantasia, Mago de Oz, Cain's Offering, e todas as outras bandas de Power Metal. 	
% \end{frame}
% %------------------------------------------------------------------------------

% %------------------------------------------------------------------------------
% \subsection{Sites legais}
% \begin{frame}{Sites legais}
%   Want to know more? See!
%   \begin{itemize}
%     \item Browse in my page \url{http://www.ezefranca.com}.
%     \item Browse on BCC page \url{http://www.sp.sustech.br/bcc}.
%     \item Thanks WriteLaTeX from Support! \url{http://www.writelatex.com}.
%   \end{itemize}
  
% \end{frame}
% %------------------------------------------------------------------------------

% -----------------------------------------------------------------------------
\end{document}
%-----------------------------------------------Este comentario nunca aparecera